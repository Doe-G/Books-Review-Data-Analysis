\section{Hito 0}

Para el Hito 0 se nos había solicitado seleccionar nuestra(s) Base(s) de Datos, junto con el Problema que buscaremos resolver, es decir, la motivación detrás de este proyecto. \newline

Las fuentes a utilizar son las siguientes:

\begin{itemize}
    \item \hyperlink{https://www.kaggle.com/datasets/ruchi798/bookcrossing-dataset}{https://www.kaggle.com/datasets/ruchi798/bookcrossing-dataset}
    \item \hyperlink{https://www.kaggle.com/datasets/mohamedbakhet/amazon-books-reviews}{https://www.kaggle.com/datasets/mohamedbakhet/amazon-books-reviews}
    \item \hyperlink{https://www.kaggle.com/datasets/jealousleopard/goodreadsbooks}{https://www.kaggle.com/datasets/jealousleopard/goodreadsbooks}
\end{itemize}

Por otro lado, teníamos contemplado resolver el problema: \newline

\textit{``Usando una base de datos de libros y reviews de los mismos, y mediante un análisis de datos, ver si es posible concluir que, para un mismo género literario, las reviews fluctuan según rango etario."} \newline

El asunto con este problema es que no hacía uso de al menos 4 entidades, por lo que no apuntaba a los objetivos del Proyecto como tal. \newline

A raíz de esto hemos creado una nueva motivación, la cual sería: \newline

\textit{``Usando una base de datos de libros y reviews de los mismos, y mediante un análisis de datos, ver si es posible concluir que, para un mismo género literario, las reviews fluctuan, en distintos rangos etarios preseleccionados, según la versión de un libro.''}